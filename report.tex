\documentclass{article}
\usepackage[a4paper]{geometry}
\geometry{
    left=3cm,
    right=2cm,
    top=3cm,
    bottom=3cm,
    headheight=15pt,
    headsep=1cm
}
\usepackage{ctex}
\usepackage{graphicx}
\usepackage{fancyhdr}
\usepackage{lastpage} % 仅保留 lastpage 宏包
\usepackage{lipsum}
\usepackage{calc}
\usepackage{ulem}
\usepackage{tikz}
\usepackage{amsmath}
\usepackage{cleveref}
\usetikzlibrary{calc}

% 设置图片路径
\graphicspath{{./img/}}

% ===== 装订线样式设置 =====
% 使用新命令定义装订线(仅左侧垂直点划线 + 竖排文字)
\newcommand{\装订线}{%
    \begin{tikzpicture}[remember picture, overlay]
        % 左侧垂直点划线(从页面顶部下方2cm到页面底部上方2cm)
        \draw[line width=1.2pt, dotted] 
                ($(current page.north west)+(0.5cm,-2cm)$) 
                -- ($(current page.west)+(0.5cm,2cm)$);
        % 下部点划线
        \draw[line width=1.2pt, dotted] 
            ($(current page.west)+(0.5cm,-2cm)$) 
            -- ($(current page.south west)+(0.5cm,2cm)$);
        
        % 竖排文字(使用rotatebox实现)
        \node[anchor=center, inner sep=0pt] 
            at ($(current page.west)+(0.5cm + 0.075cm,0)$)  {
                {%
                    \parbox{1.5cm}{\centering\small 装 \\[0.5cm] 订 \\[0.5cm] 线}
                }
            };
    \end{tikzpicture}
}

% 定义第一页页眉样式
\fancypagestyle{firstpage}{
    \fancyhf{} 
    \renewcommand{\headrulewidth}{0pt} 
    \fancyheadoffset[L]{0.5cm} 
    \fancyhead[L]{\装订线} % 使用装订线命令
}

% 定义第二页及后续页页眉样式
\pagestyle{fancy}
\fancyhf{}
\renewcommand{\headrulewidth}{0pt}
\fancyheadoffset[L]{0.5cm}
\fancyhead[L]{\装订线} % 使用装订线命令
\fancyhead[C]{
    实验名称:\uline{\makebox[4cm][c]{快乐实验不再有}}
    姓名:\uline{\makebox[4cm][c]{乐乐乐}}
    学号:\uline{\makebox[4cm][c]{哈哈哈}}
}

% 页脚显示页码
\fancyfoot[C]{\thepage/\pageref{LastPage}}% 用\totalpages获取总页数

\renewcommand{\thesection}{\chinese{section}、}
\renewcommand{\thesubsection}{\arabic{subsection}}
\begin{document}
\thispagestyle{firstpage} % 使用第一页的页眉样式
% 抬头部分
\begin{minipage}[t]{0.7\textwidth}
    \centering
    \includegraphics[width=0.6\textwidth]{logo.png} % 调整图片宽度
    \Large{\textbf{实验报告}}
\end{minipage}
\hfill
% 右侧表格区域(整体右对齐)
\begin{minipage}[t]{0.3\textwidth}
    \begin{flushright}
        \begin{tabular}{@{}ll@{}}
            专业: & \underline{\makebox[3cm][c]{哈哈哈}} \\
            姓名: & \underline{\makebox[3cm][c]{乐乐乐}}\\
            学号: & \underline{\makebox[3cm][c]{哈哈哈}}\\
            日期: & \underline{\makebox[3cm][c]{\today}}\\
            地点: &  \underline{\makebox[3cm][c]{快乐星球}}
        \end{tabular}
    \end{flushright}
\end{minipage}

\vspace{1cm}

% 课程信息区域
课程名称:\underline{\makebox[4cm][c]{快乐机理研究}} 
指导老师:\underline{\makebox[2cm][c]{乐乐}}
成绩:\underline{\makebox[4cm][c]{}}

实验名称:\underline{\makebox[4cm][c]{快乐实验不再有}} 
实验类型:\underline{\makebox[2cm][c]{}}
同组同学姓名:\underline{\makebox[2.5cm][c]{}}




% 左右分栏章节标题(仅显示,无实际内容)
\begin{minipage}[t]{0.45\textwidth}
    一、实验目的和要求(必填)

    三、主要仪器设备(必填)

    五、实验数据记录和处理

    七、讨论、心得
\end{minipage}
\hfill
\begin{minipage}[t]{0.45\textwidth}
    二、实验内容和原理(必填)

    四、操作方法和实验步骤

    六、实验结果与分析(必填)
\end{minipage}


% 实际章节内容(使用自动编号)
\setcounter{section}{0} % 重置章节计数器

\section{实验目的和要求}

\section{实验内容和原理}

\section{主要仪器设备}

\section{操作方法和实验步骤}

\section{实验数据记录和处理}

\section{实验结果与分析}

\section{讨论、心得}

\end{document}